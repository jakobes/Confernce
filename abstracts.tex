\documentclass{article}
\usepackage{graphicx}

\usepackage[utf8]{inputenc}
\usepackage[T1]{fontenc}



\newcommand{\overview}[3] 
{
#1

#2

#3
  \clearpage
}
  
  
  
\begin{document}



\overview{Cinzia Anita Maria Progida}{2D or not 2D: Dimensional and mechanical cues in cell migration.}{The movement of cells within the body, or cell migration, is fundamental for both physiological and
pathological processes including wound healing, immune response and cancer metastasis. Cell
motility involves the reorganization of the cytoskeleton and the movement of organelles, usually
triggered by external cues. Indeed, cells respond to biochemical or mechanical stimuli by activating
signaling pathways, re-organizing the cytoskeleton and generating forces. We study how subcellular
components and their intracellular dynamics affect the cellular response to extracellular stimuli such
as dimensional cues and physical confinement and the molecular mechanisms involved in
modulating cell migration.
The majority of the cell migration studies have been conducted using two-dimensional (2D) systems,
where adhesion formation and turnover represent important steps. However, cells require different
migratory strategies in 2D systems compared to the more physiological 3D systems. In addition,
some type of cells, such as immune cells, exhibit an adhesion-independent migration strategy in
their native environment.
I will present the different migration assays we use in order to characterize how subcellular
components influence the cellular response to dimensional cues and physical confinement. We
either embed cells in collagen gels or we use micro-fabricated channels where the cells are
completely confined. We also study how sub-cellular- level processes affect the forces exerted by the
cells on their environment using micropillars.
}



\overview{Liesbeth M. C. Janssen,
Theory of Polymers and Soft Matter, Department of Applied Physics, TU Eindhoven, The Netherlands}{Membrane formation and compartmentalization in synthetic active matter}{

Active matter refers to systems whose constituent agents can move autonomously through the
consumption of energy. Such materials exhibit rich non-equilibrium dynamics and provide a framework
to understand the complex collective behavior seen in many living systems. In this talk, I will highlight
recent results of particle-resolved simulations of active rods that mimic cell-like properties such as
membrane formation and compartmentalization of particles. The crucial ingredients in our model are
the particles’ intrinsic self-propulsion, interparticle aligning interactions, and an inhomogeneous
motility field. We thus show that a minimal artificial active-matter system can self-organize into
structures reminiscent of biological patterns. Our predictions may be verified experimentally in e.g.
vibrated granular matter and other dry active systems with a spatially dependent self-propulsion speed.


Reference: J. Grauer, H. Löwen, and L.M.C. Janssen, Spontaneous membrane formation and self-
encapsulation of active rods in an inhomogeneous motility field, arXiv:1707.03405.
}


\overview{Margarita Staykova}{
Mechanics of the cell interface studied by supported lipid bilayers
}{ The cell membrane undergoes complex morphological and surface area transformations while being confined to an underlying actin cortex, the membranes of neighboring cells or other extracellular structures. To understand the role of confinement in the membrane processes we adhere synthetic lipid bilayers to artificial substrates and subject them to perturbations that are common to the cell membrane - 1) substrate area changes, or 2) intake of extra lipids. Our results show that confined lipid bilayers regulate the arising changes in their lipid density either by the expulsion and absorption of lipid protrusions, such as tubes and vesicles, or by sliding over the substrate. Similar processes have been recently confirmed in living cells. We provide a theoretical framework that rationalizes the membrane behavior in terms of the membrane elasticity, and the adhesion and hydrodynamic interactions between the membrane and the substrate. 
}



\overview{Alexandra K. Diem}{The role of biomechanical modelling in discovering the brain’s lymphatic system}{How does the human brain eliminate waste products? This simple yet crucial question has
occupied medical research for decades, in particular with regards Alzheimer's disease,
whose onset is closely associated with a failure to remove the cerebral waste product
amyloid-$\beta$ (A$\beta$). Analytical and numerical modelling of the biomechanical processes
underlying waste removal from the brain can play a crucial role in evaluating hypotheses
and identifying its driving mechanisms. This talk describes the journey of attempting to
resolve some of the questions surrounding the removal of A$\beta$ in healthy individuals,
starting from the biomedical evidence, to the analytical and numerical methods to disprove
one of the most popular hypotheses and the derivation of new hypotheses. I will argue that
in order to fully resolve the onset and development of neurodegenerative diseases we
require fully interdisciplinary scientists working side-by- side with experimental researchers
and taking into account the multiple scales involved in the cerebral waste transport
processes in the human brain.}

\overview{Professor M\'ar M\'asson}{}{}


\overview{ MASSIERA Gladys}{Biophysical approach of the mucociliary function: Mucus rheology and beating coordination}{The mucociliary function of the bronchial epithelium ensures the continuous clearance of the respiratory system, which relies on two main elements: mucus and cilia beating coordination.
We perform here a rheological characterization of mucus samples extracted from ALI (Air-liquid interface) cultures of bronchial epithelium. Our approach combines  macro- and micro-rheology techniques with the aim of quantifying the mucus viscoelastic properties at different length scales (from the size of bronchial cilia up to the scale on which mucus is transported)
}




\overview{Susanne Liese}{Hydration Effects Turn a Highly Stretched Polymer from an Entropic into an Energetic Spring}{
Polyethylene glycol (PEG) is a structurally simple and nontoxic water-soluble polymer
that is widely used in medical and pharmaceutical applications as molecular linker
and spacer. In such applications, PEG’ s elastic response against conformational
deformations is key to its function. According to text-book knowledge, a polymer
reacts to the stretching of its end-to-end separation by a decrease in entropy that is
due to the reduction of available conformations, which is why polymers are commonly
called entropic springs. By a combination of single-molecule force spectroscopy
experiments with molecular dynamics simulations in explicit water, we show that
entropic hydration eff ects almost exactly compensate the chain conformational
entropy loss at high stretching. Our simulations reveal that this entropic
compensation is due to the stretching-induced release of water molecules that in the
relaxed state form double hydrogen bonds with PEG. As a consequence, the
stretching response of PEG is predominantly of energetic, not of entropic, origin at
high forces and caused by hydration eff ects, while PEG backbone deformations only
play a minor role. These fi ndings demonstrate the importance of hydration for the
mechanics of macromolecules and constitute a case example that sheds light on the
antagonistic interplay of conformational and hydration degrees of freedom.}


\overview{Klas Pettersen}{Transport of nutrients and clearance of waste products through the tiny
spaces surrounding brain cells}{Transport of nutrients and clearance of waste products are
prerequisites for healthy brain function. The brain lacks lymph vessels and must
rely on other mechanisms for clearance of waste products, including
amyloid $\beta$ that may form pathological aggregates if not effectively cleared. It is
still debated whether solutes are transported through the tiny spaces
surrounding brain cells, the interstitial space, by pressure-mediated bulk flow or
by diffusion. In a recently published article we simulated interstitial bulk flow
within 3D electron microscope reconstructions of hippocampal tissue [1]. We
found that the permeability is one to two orders of magnitudes lower than values
typically seen in the literature, arguing against bulk flow as the dominant
transport mechanism. Further, we showed that solutes of all sizes are more
easily transported through the interstitium by diffusion than by bulk flow. We
conclude that diffusion within the interstitial space combined with advection
along vessels is likely to substitute for the lymphatic drainage system in other
organs.


[1] Holter KE, Kehlet B, Devor A, Sejnowski TJ, Dale AM, Omholt SW, Ottersen OP,
Nagelhus EA, Mardal K-A and Pettersen KH (2017). Interstitial solute transport
in 3D reconstructed neuropil occurs by diffusion rather than bulk flow.
Proceedings of the National Academy of Sciences, 6, 201706942.
http://doi.org/10.1073/pnas.1706942114
}

\overview{Professor Raymond E. Goldstein}{Upside-Down and Inside-Out: The Biomechanics of Cell Sheet Folding}{}


\overview{Guillaume Salbreux}{Physics of epithelial folding}{  
Three-dimensional deformations of epithelia play a fundamental role in tissue morphogenesis. The shape of an epithelium is determined by mechanical stresses acting within the tissue cells and from the outside environment. Here we introduce a three-dimensional vertex model which allows to represent the shape of a tissue in three dimensions by a set of vertices. In the model, the motion of vertices is set by apical, lateral and basal surface and line tensions, as well as intracellular pressures and external forces. Using this framework, we discuss how patterned force generation in an epithelium can drive biological tissue folding in fold formation in the Drosophila wing disc and in pancreatic tumour formation.}


\overview{Gazzola, Mattia}{Computational design of artificial creatures}{We introduce an inverse design approach based on minimal theoretical modeling, direct numerical simulations and artificial intelligence for the investigation of animal locomotion. We will mostly focus on aquatic and terrestrial limbless creatures and discuss the identification of optimal swimming gaits and morphologies, as well as the design of cyborg creatures.}


\overview{Alask}{}{}


\overview{Rognes}{}{}


\overview{}{}{}


\overview{}{}{}


\overview{}{}{}


\overview{}{}{}
\overview{}{}{}
\overview{}{}{}

\end{document}